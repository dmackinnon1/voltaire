\vchapter{B}

\vword{BANISHMENT}


Banishment for a period or for life, punishment to which one condemns
delinquents, or those one wishes to appear as such.

Not long ago one banished outside the sphere of jurisdiction a petty
thief, a petty forger, a man guilty of an act of violence. The result
was that he became a big robber, a forger on a big scale, and murderer
within the sphere of another jurisdiction. It is as if we threw into our
neighbours' fields the stones which incommode us in our own.

Those who have written on the rights of men, have been much tormented to
know for certain if a man who has been banished from his fatherland
still belongs to his fatherland. It is nearly the same thing as asking
if a gambler who has been driven away from the gaming-table is still one
of the gamblers.

If to every man it is permitted by natural right to choose his
fatherland, he who has lost the right of citizen can, with all the more
reason, choose for himself a new fatherland; but can he bear arms
against his former fellow-citizens? There are a thousand examples of it.
How many French protestants naturalized in Holland, England and Germany
have served against France, and against armies containing their own
kindred and their own brothers! The Greeks who were in the King of
Persia's armies made war on the Greeks, their former compatriots. One
has seen the Swiss in the Dutch service fire on the Swiss in the French
service. It is still worse than to fight against those who have banished
you; for, after all, it seems less dishonest to draw the sword for
vengeance than to draw it for money.




\vword{BANKRUPTCY}


Few bankruptcies were known in France before the sixteenth century. The
great reason is that there were no bankers. Lombards, Jews lent on
security at ten per cent: trade was conducted in cash. Exchange,
remittances to foreign countries were a secret unknown to all judges.

It is not that many people were not ruined; but that was not called
\textbf{bankruptcy}; one said \textit{discomfiture}; this word is sweeter to the ear.
One used the word \textit{rupture} as did the Boulonnais; but rupture does not
sound so well.

The bankruptcies came to us from Italy, \textit{bancorotto, bancarotta,
gambarotta e la giustizia non impicar}. Every merchant had his bench
(\textit{banco}) in the place of exchange; and when he had conducted his
business badly, declared himself \textit{fallito}, and abandoned his property
to his creditors with the proviso that he retain a good part of it for
himself, be free and reputed a very upright man. There was nothing to be
said to him, his bench was broken, \textit{banco rotto, banca rotta}; he could
even, in certain towns, keep all his property and baulk his creditors,
provided he seated himself bare-bottomed on a stone in the presence of
all the merchants. This was a mild derivation of the old Roman
proverb--\textit{solvere aut in aere aut in cute}, to pay either with one's
money or one's skin. But this custom no longer exists; creditors have
preferred their money to a bankrupt's hinder parts.

In England and in some other countries, one declares oneself bankrupt in
the gazettes. The partners and creditors gather together by virtue of
this announcement which is read in the coffee-houses, and they come to
an arrangement as best they can.

As among the bankruptcies there are frequently fraudulent cases, it has
been necessary to punish them. If they are taken to court they are
everywhere regarded as theft, and the guilty are condemned to
ignominious penalties.

It is not true that in France the death penalty was decreed against
bankrupts without distinction. Simple failures involved no penalty;
fraudulent bankrupts suffered the penalty of death in the states of
Orleans, under Charles IX., and in the states of Blois in 1576, but
these edicts, renewed by Henry IV., were merely comminatory.

It is too difficult to prove that a man has dishonoured himself on
purpose, and has voluntarily ceded all his goods to his creditors in
order to cheat them. When there has been a doubt, one has been content
with putting the unfortunate man in the pillory, or with sending him to
the galleys, although ordinarily a banker makes a poor convict.

Bankrupts were very favourably treated in the last year of Louis XIV.'s
reign, and during the Regency. The sad state to which the interior of
the kingdom was reduced, the multitude of merchants who could not or
would not pay, the quantity of unsold or unsellable effects, the fear of
interrupting all commerce, obliged the government in 1715, 1716, 1718,
1721, 1722, and 1726 to suspend all proceedings against all those who
were in a state of insolvency. The discussions of these actions were
referred to the judge-consuls; this is a jurisdiction of merchants very
expert in these cases, and better constituted for going into these
commercial details than the parliaments which have always been more
occupied with the laws of the kingdom than with finance. As the state
was at that time going bankrupt, it would have been too hard to punish
the poor middle-class bankrupts.

Since then we have had eminent men, fraudulent bankrupts, but they have
not been punished.




\vword{BEAUTY}


Ask a toad what beauty is, the \textit{to kalon}? He will answer you that it is
his toad wife with two great round eyes issuing from her little head, a
wide, flat mouth, a yellow belly, a brown back. Interrogate a Guinea
negro, for him beauty is a black oily skin, deep-set eyes, a flat nose.
Interrogate the devil; he will tell you that beauty is a pair of horns,
four claws and a tail. Consult, lastly, the philosophers, they will
answer you with gibberish: they have to have something conforming to the
arch-type of beauty in essence, to the \textit{to kalon}.

One day I was at a tragedy near by a philosopher. "How beautiful that
is!" he said.

"What do you find beautiful there?" I asked.

"It is beautiful," he answered, "because the author has reached his
goal."

The following day he took some medicine which did him good. "The
medicine has reached its goal," I said to him. "What a beautiful
medicine!" He grasped that one cannot say a medicine is beautiful, and
that to give the name of "beauty" to something, the thing must cause you
to admire it and give you pleasure. He agreed that the tragedy had
inspired these sentiments in him, and that there was the \textit{to kalon}, beauty.

We journeyed to England: the same piece, perfectly translated, was
played there; it made everybody in the audience yawn. "Ho, ho!" he said,
\enquote{the \textit{to kalon} is not the same for the English and the French.} After
much reflection he came to the conclusion that beauty is often very
relative, just as what is decent in Japan is indecent in Rome, and what
is fashionable in Paris, is not fashionable in Pekin; and he saved
himself the trouble of composing a long treatise on beauty.

There are actions which the whole world finds beautiful. Two of Caesar's
officers, mortal enemies, send each other a challenge, not as to who
shall shed the other's blood with tierce and quarte behind a thicket as
with us, but as to who shall best defend the Roman camp, which the
Barbarians are about to attack. One of them, having repulsed the enemy,
is near succumbing; the other rushes to his aid, saves his life, and
completes the victory.

A friend sacrifices his life for his friend; a son for his father....
The Algonquin, the Frenchman, the Chinaman, will all say that that is
very \textit{beautiful} that these actions give them pleasure, that they
admire them.

They will say as much of the great moral maxims, of Zarathustra's--\enquote{In
doubt if an action be just, abstain...}; of Confucius'-- \enquote{Forget
injuries, never forget kindnesses.}

The negro with the round eyes and flat nose, who will not give the name
of \enquote{beauties} to the ladies of our courts, will without hesitation give
it to these actions and these maxims. The wicked man even will recognize
the beauty of these virtues which he dare not imitate. The beauty which
strikes the senses merely, the imagination, and that which is called
\enquote{intelligence,} is often uncertain therefore. The beauty which speaks to
the heart is not that. You will find a host of people who will tell you
that they have found nothing beautiful in three-quarters of the Iliad;
but nobody will deny that Codrus' devotion to his people was very
beautiful, supposing it to be true.

There are many other reasons which determine me not to write a treatise
on beauty.




\vword{BISHOP}


Samuel Ornik, native of Basle, was, as you know, a very amiable young
man who, besides, knew his New Testament by heart in Greek and German.
When he was twenty his parents sent him on a journey. He was charged to
carry some books to the coadjutor of Paris, at the time of the Fronde.
He arrived at the door of the archbishop's residence; the Swiss told him
that Monseigneur saw nobody. "Comrade," said Ornik to him, "you are very
rude to your compatriots. The apostles let everyone approach, and Jesus
Christ desired that people should suffer all the little children to come
to him. I have nothing to ask of your master; on the contrary, I have
brought him something."

"Come inside, then," said the Swiss.

He waits an hour in a first antechamber. As he was very naïve, he began
a conversation with a servant, who was very fond of telling all he knew
of his master. "He must be mightily rich," said Ornik, "to have this
crowd of pages and flunkeys whom I see running about the house."

"I don't know what his income is," answered the other, "but I heard it
said to Joly and the Abbé Charier that he already had two millions of
debts."

"But who is that lady coming out of the room?"

"That is Madame de Pomereu, one of his mistresses."

"She is really very pretty; but I have not read that the apostles had
such company in their bedrooms in the mornings. Ah! I think the
archbishop is going to give audience."

"Say--'His Highness, Monseigneur.'"

"Willingly." Ornik salutes His Highness, presents his books, and is
received with a very gracious smile. The archbishop says four words to
him, then climbs into his coach, escorted by fifty horsemen. In
climbing, Monseigneur lets a sheath fall. Ornik is quite astonished that
Monseigneur carries so large an ink-horn in his pocket. "Don't you see
that's his dagger?" says the chatterbox. "Everyone carries a dagger when
he goes to parliament."

"That's a pleasant way of officiating," says Ornik; and he goes away
very astonished.

He traverses France, and enlightens himself from town to town; thence he
passes into Italy. When he is in the Pope's territory, he meets one of
those bishops with a thousand crowns income, walking on foot. Ornik was
very polite; he offers him a place in his cambiature. "You are doubtless
on your way to comfort some sick man, Monseigneur?"

"Sir, I am on my way to my master's."

"Your master? that is Jesus Christ, doubtless?"

"Sir, it is Cardinal Azolin; I am his almoner. He pays me very poorly;
but he has promised to place me in the service of Donna Olimpia, the
favourite sister-in-law di nostro signore."

"What! you are in the pay of a cardinal? But do you not know that there
were no cardinals in the time of Jesus Christ and St. John?"

"Is it possible?" cried the Italian prelate.

"Nothing is more true; you have read it in the Gospel."

"I have never read it," answered the bishop; "all I know is Our Lady's
office."

"I tell you there were neither cardinals nor bishops, and when there
were bishops, the priests were their equals almost, according to
Jerome's assertions in several places."

"Holy Virgin," said the Italian. "I knew nothing about it: and the
popes?"

"There were not any popes any more than cardinals."

The good bishop crossed himself; he thought he was with an evil spirit,
and jumped out of the cambiature.




\vword{BOOKS}


You despise them, books, you whose whole life is plunged in the vanities
of ambition and in the search for pleasure or in idleness; but think
that the whole of the known universe, with the exception of the savage
races is governed by books alone. The whole of Africa right to Ethiopia
and Nigritia obeys the book of the Alcoran, after having staggered under
the book of the Gospel. China is ruled by the moral book of Confucius; a
greater part of India by the book of the Veidam. Persia was governed for
centuries by the books of one of the Zarathustras.

If you have a law-suit, your goods, your honour, your life even depends
on the interpretation of a book which you never read.

\textbf{Robert the Devil}, the \textbf{Four Sons of Aymon}, the \textbf{Imaginings of Mr.
Oufle}, are books also; but it is with books as with men; the very small
number play a great part, the rest are mingled in the crowd.

Who leads the human race in civilized countries? those who know how to
read and write. You do not know either Hippocrates, Boerhaave or
Sydenham; but you put your body in the hands of those who have read
them. You abandon your soul to those who are paid to read the Bible,
although there are not fifty among them who have read it in its entirety
with care.

To such an extent do books govern the world, that those who command
to-day in the city of the Scipios and the Catos have desired that the
books of their law should be only for them; it is their sceptre; they
have made it a crime of \textit{lèse-majesté} for their subjects to look there
without express permission. In other countries it has been forbidden to
think in writing without letters patent.

There are nations among whom thought is regarded purely as an object of
commerce. The operations of the human mind are valued there only at two
sous the sheet.

In another country, the liberty of explaining oneself by books is one of
the most inviolable prerogatives. Print all that you like under pain of
boring or of being punished if you abuse too considerably your natural
right.

Before the admirable invention of printing, books were rarer and more
expensive than precious stones. Almost no books among the barbarian
nations until Charlemagne, and from him to the French king Charles V.,
surnamed "the wise"; and from this Charles right to François Ier, there
is an extreme dearth.

The Arabs alone had books from the eighth century of our era to the
thirteenth.

China was filled with them when we did not know how to read or write.

Copyists were much employed in the Roman Empire from the time of the
Scipios up to the inundation of the barbarians.

The Greeks occupied themselves much in transcribing towards the time of
Amyntas, Philip and Alexander; they continued this craft especially in
Alexandria.

This craft is somewhat ungrateful. The merchants always paid the authors
and the copyists very badly. It took two years of assiduous labour for a
copyist to transcribe the Bible well on vellum. What time and what
trouble for copying correctly in Greek and Latin the works of Origen, of
Clement of Alexandria, and of all those other authors called "fathers."

The poems of Homer were long so little known that Pisistratus was the
first who put them in order, and who had them transcribed in Athens,
about five hundred years before the era of which we are making use.

To-day there are not perhaps a dozen copies of the Veidam and the
Zend-Avesta in the whole of the East.

You would not have found a single book in the whole of Russia in 1700,
with the exception of Missals and a few Bibles in the homes of aged men
drunk on brandy.

To-day people complain of a surfeit: but it is not for readers to
complain; the remedy is easy; nothing forces them to read. It is not any
the more for authors to complain. Those who make the crowd must not cry
that they are being crushed. Despite the enormous quantity of books, how
few people read! and if one read profitably, one would see the
deplorable follies to which the common people offer themselves as prey
every day.

What multiplies books, despite the law of not multiplying beings
unnecessarily, is that with books one makes others; it is with several
volumes already printed that a new history of France or Spain is
fabricated, without adding anything new. All dictionaries are made with
dictionaries; almost all new geography books are repetitions of
geography books. The Summation of St. Thomas has produced two thousand
fat volumes of theology; and the same family of little worms that have
gnawed the mother, gnaw likewise the children.




\vword{BOULEVERD OR BOULEVART}


Boulevart, fortification, rampart. Belgrade is the boulevart of the
Ottoman Empire on the Hungarian side. Who would believe that this word
originally signified only a game of bowls? The people of Paris played
bowls on the grass of the rampart; this grass was called the \textit{verd},
like the grass market. \textit{On boulait sur le verd.} From there it comes
that the English, whose language is a copy of ours in almost all the
words which are not Saxon, have called the game of bowls
"bowling-green," the \textit{verd} (green) of the game of bowls. We have taken
back from them what we had lent them. Following their example, we gave
the name of \textit{boulingrins}, without knowing the strength of the word, to
the grass-plots we introduced into our gardens.

I once heard two good dames who were going for a walk on the
\textit{Bouleverd}, and not on the \textit{Boulevart}. People laughed at them, and
wrongly. But in all matters custom carries the day; and everyone who is
right against custom is hissed or condemned.




\vword{BOURGES}


Our questions barely turn on geography; but let us be permitted to mark
in two words our astonishment about the town of Bourges. The
"Dictionnaire de Trévoux" claims that "it is one of the most ancient
towns of Europe, that it was the seat of the empire of the Gauls, and
gave kings to the Celts."

I do not wish to combat the ancientness of any town or any family. But
was there ever an empire of the Gauls? Did the Celts have kings? This
mania for antiquity is a malady from which one will not be healed so
soon. The Gauls, Germany, Scandinavia have nothing that is antique save
the land, the trees and the animals. If you want antiquities, go toward
Asia, and even then it is very small beer. Man is ancient and monuments
new, that is what we have in view in more than one article.

If it were a real benefit to be born in a stone or wooden enclosure more
ancient than another, it would be very reasonable to make the foundation
of one's town date back to the time of the war of the giants; but since
there is not the least advantage in this vanity, one must break away
from it. That is all I had to say about Bourges.




\vword{BRAHMINS}


Is it not probable that the Brahmins were the first legislators of the
earth, the first philosophers, the first theologians?

Do not the few monuments of ancient history which remain to us form a
great presumption in their favour, since the first Greek philosophers
went to them to learn mathematics, and since the most ancient
curiosities collected by the emperors of China are all Indian?

We will speak elsewhere of the "Shasta"; it is the first book of
theology of the Brahmins, written about fifteen hundred years before
their "Veidam," and anterior to all the other books.

Their annals make no mention of any war undertaken by them at any time.
The words for \textit{arms}, to \textit{kill}, to \textit{maim}, are not to be found either
in the fragments of the "Shasta" which we have, or in the "Ezourveidam,"
or in the "Cormoveidam." I can at least give the assurance that I did
not see them in these last two collections: and what is still more
singular is that the "Shasta" which speaks of a conspiracy in heaven,
makes no mention of any war in the great peninsula enclosed between the
Indus and the Ganges.

The Hebrews, who were known so late, never name the Brahmins; they had
no knowledge of India until after the conquests of Alexander, and their
settling in Egypt, of which they had said so much evil. The name of
India is to be found only in the Book of Esther, and in that of Job
which was not Hebrew. One remarks a singular contrast between the sacred
books of the Hebrews, and those of the Indians. The Indian books
announce only peace and gentleness; they forbid the killing of animals:
the Hebrew books speak only of killing, of the massacre of men and
beasts; everything is slaughtered in the name of the Lord; it is quite
another order of things.

It is incontestably from the Brahmins that we hold the idea of the fall
of the celestial beings in revolt against the Sovereign of nature; and
it is from there probably that the Greeks drew the fable of the Titans.
It is there also that the Jews at last took the idea of the revolt of
Lucifer, in the first century of our era.

How could these Indians suppose a revolt in heaven without having seen
one on earth? Such a jump from human nature to divine nature is barely
conceivable. Usually one goes from known to unknown.

One does not imagine a war of giants until one has seen some men more
robust than the others tyrannize over their fellows. The first Brahmins
must either have experienced violent discords, or at least have seen
them in heaven.

It is a very astonishing phenomenon for a society of men who have never
made war to have invented a species of war made in the imaginary spaces,
or in a globe distant from ours, or in what is called the "firmament,"
the "empyrean." But it must be carefully observed that in this revolt of
celestial beings against their Sovereign no blows were struck, no
celestial blood flowed, no mountains hurled at the head, no angels cut
in two, as in Milton's sublime and grotesque poem.

According to the "Shasta," it is only a formal disobedience to the
orders of the Most High, a cabal which God punishes by relegating the
rebellious angels to a vast place of shadows called "Ondera" during the
period of an entire mononthour. A mononthour is four hundred and
twenty-six millions of our years. But God deigned to pardon the guilty
after five thousand years, and their ondera was only a purgatory.

He made "Mhurd" of them, men, and placed them in our globe on condition
that they should not eat animals, and that they should not copulate with
the males of their new species, under pain of returning to ondera.

Those are the principal articles of the Brahmins' faith, which have
lasted without interruption from immemorial times right to our day: it
seems strange to us that among them it should be as grave a sin to eat a
chicken as to commit sodomy.

This is only a small part of the ancient cosmogony of the Brahmins.
Their rites, their pagodas, prove that among them everything was
allegorical; they still represent virtue beneath the emblem of a woman
who has ten arms, and who combats ten mortal sins represented by
monsters. Our missionaries have not failed to take this image of virtue
for that of the devil, and to assure us that the devil is worshipped in
India. We have never been among these people but to enrich ourselves and
to calumniate them.

Really we have forgotten a very essential thing in this little article
on the Brahmins; it is that their sacred books are filled with
contradictions. But the people do not know of them, and the doctors have
solutions ready, figurative meanings, allegories, symbols, express
declarations of Birma, Brahma and Vitsnou, which should close the mouths
of all who reason.

