\vchapter{A}

\vword{ADULTERY}


NOTE ON A MAGISTRATE WRITTEN ABOUT 1764







A senior magistrate of a French town had the misfortune to have a wife
who was debauched by a priest before her marriage, and who since covered
herself with disgrace by public scandals: he was so moderate as to leave
her without noise. This man, about forty years old, vigorous and of
agreeable appearance, needs a woman; he is too scrupulous to seek to
seduce another man's wife, he fears intercourse with a public woman or
with a widow who would serve him as concubine. In this disquieting and
sad state, he addresses to his Church a plea of which the following is a
précis:

My wife is criminal, and it is I who am punished. Another woman is
necessary as a comfort to my life, to my virtue even; and the sect of
which I am a member refuses her to me; it forbids me to marry an honest
girl. The civil laws of to-day, unfortunately founded on canon law,
deprive me of the rights of humanity. The Church reduces me to seeking
either the pleasures it reproves, or the shameful compensations it
condemns; it tries to force me to be criminal.

I cast my eyes over all the peoples of the earth; there is not a single
one except the Roman Catholic people among whom divorce and a new
marriage are not natural rights.

What upheaval of the rule has therefore made among the Catholics a
virtue of undergoing adultery, and a duty of lacking a wife when one has
been infamously outraged by one's own?

Why is a bond that has rotted indissoluble in spite of the great law
adopted by the code, \textit{quidquid ligatur dissolubile est}? I am allowed a
separation \textit{a mensa et thoro}, and I am not allowed divorce. The law
can deprive me of my wife, and it leaves me a name called \enquote{sacrament}!
What a contradiction! what slavery! and under what laws did we receive
birth!

What is still more strange is that this law of my Church is directly
contrary to the words which this Church itself believes to have been
uttered by Jesus Christ: \enquote{Whosoever shall put away his wife, except it
be for fornication, and shall marry another, committeth adultery} (Matt.
xix. 9).

I do not examine whether the pontiffs of Rome are in the right to
violate at their pleasure the law of him they regard as their master;
whether when a state has need of an heir, it is permissible to repudiate
her who can give it one. I do not inquire if a turbulent woman,
demented, homicidal, a poisoner, should not be repudiated equally with
an adulteress: I limit myself to the sad state which concerns me: God
permits me to remarry, and the Bishop of Rome does not permit me.

Divorce was a practice among Catholics under all the emperors; it was
also in all the dismembered states of the Roman Empire. The kings of
France, those called \enquote{of the first line,} almost all repudiated their
wives in order to take new ones. At last came Gregory IX., enemy of the
emperors and kings, who by a decree made marriage an unshakeable yoke;
his decretal became the law of Europe. When the kings wanted to
repudiate a wife who was an adulteress according to Jesus Christ's law,
they could not succeed; it was necessary to find ridiculous pretexts.
Louis the younger was obliged, to accomplish his unfortunate divorce
from Eleanor of Guienne, to allege a relationship which did not exist.
Henry IV., to repudiate Marguerite de Valois, pretexted a still more
false cause, a refusal of consent. One had to lie to obtain a divorce
legitimately.

What! a king can abdicate his crown, and without the Pope's permission
he cannot abdicate his wife! Is it possible that otherwise enlightened
men have wallowed so long in this absurd servitude!

That our priests, that our monks renounce wives, to that I consent; it
is an outrage against population, it is a misfortune for them, but they
merit this misfortune which they have made for themselves. They have
been the victims of the popes who wanted to have in them slaves,
soldiers without families and without fatherland, living solely for the
Church: but I, magistrate, who serve the state all day, I need a wife in
the evening; and the Church has not the right to deprive me of a benefit
which God accords me. The apostles were married, Joseph was married, and
I want to be. If I, Alsacian, am dependent on a priest who dwells at
Rome, if this priest has the barbarous power to rob me of a wife, let
him make a eunuch of me for the singing of \vref{Misereres} in his chapel.

\noindent
NOTE FOR WOMEN

Equity demands that, having recorded this note in favour of husbands, we
should also put before the public the case in favour of wives, presented
to the junta of Portugal by a Countess of Arcira. This is the substance
of it:

The Gospel has forbidden adultery for my husband just as for me; he will
be damned as I shall, nothing is better established. When he committed
twenty infidelities, when he gave my necklace to one of my rivals, and
my ear-rings to another, I did not ask the judges to have him shaved, to
shut him up among monks and to give me his property. And I, for having
imitated him once, for having done with the most handsome young man in
Lisbon what he did every day with impunity with the most idiotic
strumpets of the court and the town, have to answer at the bar before
licentiates each of whom would be at my feet if we were alone together
in my closet; have to endure at the court the usher cutting off my hair
which is the most beautiful in the world; and being shut up among nuns
who have no common sense, deprived of my dowry and my marriage
covenants, with all my property given to my coxcomb of a husband to help
him seduce other women and to commit fresh adulteries.

I ask if it is just, and if it is not evident that the laws were made by
cuckolds?

In answer to my plea I am told that I should be happy not to be stoned
at the city gate by the canons, the priests of the parish and the whole
populace. This was the practice among the first nation of the earth, the
chosen nation, the cherished nation, the only one which was right when
all the others were wrong.

To these barbarities I reply that when the poor adulteress was presented
by her accusers to the Master of the old and new law, He did not have
her stoned; that on the contrary He reproached them with their
injustice, that he laughed at them by writing on the ground with his
finger, that he quoted the old Hebraic proverb--\enquote{He that is without sin
among you, let him first cast a stone at her}; that then they all
retired, the oldest fleeing first, because the older they were the more
adulteries had they committed.

The doctors of canon law answer me that this history of the adulteress
is related only in the Gospel of St. John, that it was not inserted
there until later. Leontius, Maldonat, affirm that it is not to be found
in a single ancient Greek copy; that none of the twenty-three early
commentators mentions it. Origen, St. Jerome, St. John Chrysostom,
Theophilact, Nonnus, do not recognize it at all. It is not to be found
in the Syriac Bible, it is not in Ulphilas' version.

That is what my husband's advocates say, they who would have me not only
shaved, but also stoned.

But the advocates who pleaded for me say that Ammonius, author of the
third century, recognized this story as true, and that if St. Jerome
rejects it in some places, he adopts it in others; that, in a word, it
is authentic to-day. I leave there, and I say to my husband: \enquote{If you are
without sin, shave me, imprison me, take my property; but if you have
committed more sins than I have, it is for me to shave you, to have you
imprisoned, and to seize your fortune. In justice these things should be
equal.}

My husband answers that he is my superior and my chief, that he is more
than an inch taller, that he is shaggy as a bear; that consequently I
owe him everything, and that he owes me nothing.

But I ask if Queen Anne of England is not her husband's chief? if her
husband the Prince of Denmark, who is her High Admiral, does not owe her
entire obedience? and if she would not have him condemned by the court
of peers if the little man's infidelity were in question? It is
therefore clear that if the women do not have the men punished, it is
when they are not the stronger.
\vword{ADVOCATE}


An advocate is a man who, not having a sufficient fortune to buy one of
those resplendent offices on which the universe has its eyes, studies
the laws of Theodosius and Justinian for three years, so that he may
learn the usages of Paris, and who finally, being registered, has the
right to plead causes for money, if he have a strong voice.




\vword{ANCIENTS AND MODERNS}


The great dispute between the ancients and the moderns is not yet
settled; it has been on the table since the silver age succeeded the
golden age. Mankind has always maintained that the good old times were
much better than the present day. Nestor, in the \enquote{Iliad,} wishing to
insinuate himself as a wise conciliator into the minds of Achilles and
Agamemnon, starts by saying to them--\enquote{I lived formerly with better men
than you; no, I have never seen and I shall never see such great
personages as Dryas, Cenæus, Exadius, Polyphemus equal to the gods,
etc.}

Posterity has well avenged Achilles for Nestor's poor compliment. Nobody
knows Dryas any longer; one has hardly heard speak of Exadius, or of
Cenæus; and as for Polyphemus equal to the gods, he has not too good a
reputation, unless the possession of a big eye in one's forehead, and
the eating of men raw, are to have something of the divine.

Lucretius does not hesitate to say that nature has degenerated (lib. II.
v. 1159). Antiquity is full of eulogies of another more remote
antiquity. Horace combats this prejudice with as much finesse as force
in his beautiful Epistle to Augustus (Epist. I. liv. ii.). \enquote{Must our
poems, then,} he says, \enquote{be like our wines, of which the oldest are
always preferred?}

The learned and ingenious Fontenelle expresses himself on this subject
as follows:

\enquote{The whole question of the pre-eminence between the ancients and the
moderns, once it is well understood, is reduced to knowing whether the
trees which formerly were in our countryside were bigger than those of
to-day. In the event that they were, Homer, Plato, Demosthenes cannot
be equalled in these latter centuries.}

\enquote{Let us throw light on this paradox. If the ancients had more intellect
than us, it is that the brains of those times were better ordered,
formed of firmer or more delicate fibres, filled with more animal
spirits; but in virtue of what were the brains of those times better
ordered? The trees also would have been bigger and more beautiful; for
if nature was then younger and more vigorous, the trees, as well as
men's brains, would have been conscious of this vigour and this youth.}
(\enquote{Digression on the Ancients and the Moderns,} vol. 4, 1742 edition.)

With the illustrious academician's permission, that is not at all the
state of the question. It is not a matter of knowing whether nature has
been able to produce in our day as great geniuses and as good works as
those of Greek and Latin antiquity; but to know whether we have them in
fact. Without a doubt it is not impossible for there to be as big oaks
in the forest of Chantilli as in the forest of Dodona; but supposing
that the oaks of Dodona had spoken, it would be quite clear that they
had a great advantage over ours, which in all probability will never
speak.

Nature is not bizarre; but it is possible that she gave the Athenians a
country and a sky more suitable than Westphalia and the Limousin for
forming certain geniuses. Further, it is possible that the government of
Athens, by seconding the climate, put into Demosthenes' head something
that the air of Climart and La Grenouillère and the government of
Cardinal de Richelieu did not put into the heads of Omer Talon and
Jérome Bignon.

This dispute is therefore a question of fact. Was antiquity more fecund
in great monuments of all kinds, up to the time of Plutarch, than modern
centuries have been from the century of the Medicis up to Louis XIV.
inclusive?

The Chinese, more than two hundred years before our era, constructed
that great wall which was not able to save them from the invasion of the
Tartars. The Egyptians, three thousand years before, had overloaded the
earth with their astonishing pyramids, which had a base of about ninety
thousand square feet. Nobody doubts that, if one wished to undertake
to-day these useless works, one could easily succeed by a lavish
expenditure of money. The great wall of China is a monument to fear; the
pyramids are monuments to vanity and superstition. Both bear witness to
a great patience in the peoples, but to no superior genius. Neither the
Chinese nor the Egyptians would have been able to make even a statue
such as those which our sculptors form to-day.

The chevalier Temple, who has made it his business to disparage all the
moderns, claims that in architecture they have nothing comparable to the
temples of Greece and Rome: but, for all that he is English, he must
agree that the Church of St. Peter is incomparably more beautiful than
the Capitol was.

It is curious with what assurance he maintains that there is nothing new
in our astronomy, nothing in the knowledge of the human body, unless
perhaps, he says, the circulation of the blood. Love of his own opinion,
founded on his vast self-esteem, makes him forget the discovery of the
satellites of Jupiter, of the five moons and the ring of Saturn, of the
rotation of the sun on its axis, of the calculated position of three
thousand stars, of the laws given by Kepler and Newton for the heavenly
orbs, of the causes of the precession of the equinoxes, and of a hundred
other pieces of knowledge of which the ancients did not suspect even the
possibility.

The discoveries in anatomy are as great in number. A new universe in
little, discovered by the microscope, was counted for nothing by the
chevalier Temple; he closed his eyes to the marvels of his
contemporaries, and opened them only to admire ancient ignorance.

He goes so far as to pity us for having nothing left of the magic of the
Indians, the Chaldeans, the Egyptians; and by this magic he understands
a profound knowledge of nature, whereby they produced miracles: but he
does not cite one miracle, because in fact there never were any. \enquote{What
has become,} he asks, \enquote{of the charms of that music which so often
enchanted man and beast, the fishes, the birds, the snakes, and changed
their nature?}

This enemy of his century really believes the fable of Orpheus, and has
not apparently heard either the beautiful music of Italy, or even that
of France, which in truth does not charm snakes, but does charm the ears
of connoisseurs.

What is still more strange is that, having all his life cultivated
belles-lettres, he does not reason better about our good authors than
about our philosophers. He looks on Rabelais as a great man. He cites
the \enquote{Amours des Gaules} as one of our best works. He was, however, a
scholar, a courtier, a man of much wit, an ambassador, a man who had
reflected profoundly on all he had seen. He possessed great knowledge: a
prejudice sufficed to spoil all this merit.

There are beauties in Euripides, and in Sophocles still more; but they
have many more defects. One dares say that the beautiful scenes of
Corneille and the touching tragedies of Racine surpass the tragedies of
Sophocles and Euripides as much as these two Greeks surpass Thespis.
Racine was quite conscious of his great superiority over Euripides; but
he praised the Greek poet in order to humiliate Perrault.

Molière, in his good pieces, is as superior to the pure but cold
Terence, and to the droll Aristophanes, as to Dancourt the buffoon.

There are therefore spheres in which the moderns are far superior to the
ancients, and others, very few in number, in which we are their
inferiors. It is to this that the whole dispute is reduced.




\vword{ANIMALS}


What a pitiful, what a sorry thing to have said that animals are
machines bereft of understanding and feeling, which perform their
operations always in the same way, which learn nothing, perfect nothing,
etc.!

What! that bird which makes its nest in a semi-circle when it is
attaching it to a wall, which builds it in a quarter circle when it is
in an angle, and in a circle upon a tree; that bird acts always in the
same way? That hunting-dog which you have disciplined for three months,
does it not know more at the end of this time than it knew before your
lessons? Does the canary to which you teach a tune repeat it at once? do
you not spend a considerable time in teaching it? have you not seen that
it has made a mistake and that it corrects itself?

Is it because I speak to you, that you judge that I have feeling,
memory, ideas? Well, I do not speak to you; you see me going home
looking disconsolate, seeking a paper anxiously, opening the desk where
I remember having shut it, finding it, reading it joyfully. You judge
that I have experienced the feeling of distress and that of pleasure,
that I have memory and understanding.

Bring the same judgment to bear on this dog which has lost its master,
which has sought him on every road with sorrowful cries, which enters
the house agitated, uneasy, which goes down the stairs, up the stairs,
from room to room, which at last finds in his study the master it loves,
and which shows him its joy by its cries of delight, by its leaps, by
its caresses.

Barbarians seize this dog, which in friendship surpasses man so
prodigiously; they nail it on a table, and they dissect it alive in
order to show the mesenteric veins. You discover in it all the same
organs of feeling that are in yourself. Answer me, machinist, has nature
arranged all the means of feeling in this animal, so that it may not
feel? has it nerves in order to be impassible? Do not suppose this
impertinent contradiction in nature.

But the schoolmasters ask what the soul of animals is? I do not
understand this question. A tree has the faculty of receiving in its
fibres its sap which circulates, of unfolding the buds of its leaves and
its fruit; will you ask what the soul of this tree is? it has received
these gifts; the animal has received those of feeling, of memory, of a
certain number of ideas. Who has bestowed these gifts? who has given
these faculties? He who has made the grass of the fields to grow, and
who makes the earth gravitate toward the sun.

\enquote{Animals' souls are substantial forms,} said Aristotle, and after
Aristotle, the Arab school, and after the Arab school, the angelical
school, and after the angelical school, the Sorbonne, and after the
Sorbonne, nobody at all.

\enquote{Animals' souls are material,} cry other philosophers. These have not
been in any better fortune than the others. In vain have they been asked
what a material soul is; they have to admit that it is matter which has
sensation: but what has given it this sensation? It is a material soul,
that is to say that it is matter which gives sensation to matter; they
cannot issue from this circle.

Listen to other brutes reasoning about the brutes; their soul is a
spiritual soul which dies with the body; but what proof have you of it?
what idea have you of this spiritual soul, which, in truth, has feeling,
memory, and its measure of ideas and ingenuity; but which will never be
able to know what a child of six knows? On what ground do you imagine
that this being, which is not body, dies with the body? The greatest
fools are those who have advanced that this soul is neither body nor
spirit. There is a fine system. By spirit we can understand only some
unknown thing which is not body. Thus these gentlemen's system comes
back to this, that the animals' soul is a substance which is neither
body nor something which is not body.

Whence can come so many contradictory errors? From the habit men have
always had of examining what a thing is, before knowing if it exists.
The clapper, the valve of a bellows, is called in French the \enquote{soul} of a
bellows. What is this soul? It is a name that I have given to this valve
which falls, lets air enter, rises again, and thrusts it through a pipe,
when I make the bellows move.

There is not there a distinct soul in the machine: but what makes
animals' bellows move? I have already told you, what makes the stars
move. The philosopher who said, \enquote{\textit{Deus est anima brutorum}} was right;
but he should go further.




\vword{ANTIQUITY}


Have you sometimes seen in a village Pierre Aoudri and his wife
Peronelle wishing to go before their neighbours in the procession? \enquote{Our
grandfathers,} they say, \enquote{were tolling the bells before those who jostle
us to-day owned even a pig-sty.}

The vanity of Pierre Aoudri, his wife and his neighbours, knows nothing
more about it. Their minds kindle. The quarrel is important; honour is
in question. Proofs are necessary. A scholar who sings in the choir,
discovers an old rusty iron pot, marked with an \enquote{A,} first letter of the
name of the potter who made the pot. Pierre Aoudri persuades himself
that it was his ancestors' helmet. In this way was Cæsar descended from
a hero and from the goddess Venus. Such is the history of nations; such
is, within very small margins, the knowledge of early antiquity.

The scholars of Armenia \textit{demonstrate} that the terrestrial paradise was
in their land. Some profound Swedes \textit{demonstrate} that it was near Lake
Vener which is visibly a remnant of it. Some Spaniards \textit{demonstrate}
also that it was in Castille; while the Japanese, the Chinese, the
Indians, the Africans, the Americans are not sufficiently unfortunate to
know even that there was formerly a terrestrial paradise at the source
of the Phison, the Gehon, the Tigris and the Euphrates, or, if you
prefer it, at the source of the Guadalquivir, the Guadiana, the Douro
and the Ebro; for from Phison one easily makes Phaetis; and from Phaetis
one makes the Baetis which is the Guadalquivir. The Gehon is obviously
the Guadiana, which begins with a \enquote{G.} The Ebro, which is in Catalonia,
is incontestably the Euphrates, of which the initial letter is \enquote{E.}

But a Scotsman appears who \textit{demonstrates} in his turn that the garden of
Eden was at Edinburgh, which has retained its name; and it is to be
believed that in a few centuries this opinion will make its fortune.

The whole globe was burned once upon a time, says a man versed in
ancient and modern history; for I read in a newspaper that some
absolutely black charcoal has been found in Germany at a depth of a
hundred feet, between mountains covered with wood. And it is suspected
even that there were charcoal burners in this place.

Phaeton's adventure makes it clear that everything has boiled right to
the bottom of the sea. The sulphur of Mount Vesuvius proves invincibly
that the banks of the Rhine, Danube, Ganges, Nile and the great Yellow
River are merely sulphur, nitre and Guiac oil, which only await the
moment of the explosion to reduce the earth to ashes, as it has already
been. The sand on which we walk is evident proof that the earth has been
vitrified, and that our globe is really only a glass ball, just as are
our ideas.

But if fire has changed our globe, water has produced still finer
revolutions. For you see clearly that the sea, the tides of which mount
as high as eight feet in our climate, has produced mountains of a height
of sixteen to seventeen thousand feet. This is so true that some learned
men who have never been in Switzerland have found a big ship with all
its rigging petrified on Mount St. Gothard, or at the bottom of a
precipice, one knows not where; but it is quite certain that it was
there. Therefore men were originally fish, \textit{quod erat demonstrandum}.

To descend to a less antique antiquity, let us speak of the times when
the greater part of the barbarous nations left their countries, to go to
seek others which were hardly any better. It is true, if there be
anything true in ancient history, that there were some Gaulish brigands
who went to pillage Rome in the time of Camillus. Other Gaulish brigands
had passed, it is said, through Illyria on the way to hire their
services as murderers to other murderers, in the direction of Thrace;
they exchanged their blood for bread, and later established themselves
in Galatia. But who were these Gauls? were they Berichons and Angevins?
They were without a doubt Gauls whom the Romans called Cisalpines, and
whom we call Transalpines, famished mountain-dwellers, neighbours of the
Alps and the Apennines. The Gauls of the Seine and the Marne did not
know at that time that Rome existed, and could not take it into their
heads to pass Mount Cenis, as Hannibal did later, to go to steal the
wardrobes of Roman senators who at that time for all furniture had a
robe of poor grey stuff, ornamented with a band the colour of ox blood;
two little pummels of ivory, or rather dog's bone, on the arms of a
wooden chair; and in their kitchens a piece of rancid bacon.

The Gauls, who were dying of hunger, not finding anything to eat in
Rome, went off therefore to seek their fortune farther away, as was the
practice of the Romans later, when they ravaged so many countries one
after the other; as did the peoples of the North when they destroyed the
Roman Empire.

And, further, what is it which instructs very feebly about these
emigrations? It is a few lines that the Romans wrote at hazard; because
for the Celts, the Velches or the Gauls, these men who it is desired to
make pass for eloquent, at that time did not know, they and their bards,
how either to read or write.

But to infer from that that the Gauls or Celts, conquered after by a few
of Caesar's legions, and by a horde of Bourguignons, and lastly by a
horde of Sicamores, under one Clodovic, had previously subjugated the
whole world, and given their names and laws to Asia, seems to me to be
very strange: the thing is not mathematically impossible, and if it be
\textit{demonstrated}, I give way; it would be very uncivil to refuse to the
Velches what one accords to the Tartars.




\vword{ARTS}

\noindent
THAT THE NEWNESS OF THE ARTS IN NO WISE PROVES THE NEWNESS OF THE GLOBE


All the philosophers thought matter eternal but the arts appear new.
There is not one, even to the art of making bread, which is not recent.
The first Romans ate pap; and these conquerors of so many nations never
thought of either windmills or watermills. This truth seems at first to
contradict the antiquity of the globe such as it is, or supposes
terrible revolutions in this globe. The inundations of barbarians can
hardly annihilate arts which have become necessary. I suppose that an
army of negroes come among us like locusts, from the mountains of
Cobonas, through the Monomotapa, the Monoemugi, the Nosseguais, the
Maracates; that they have traversed Abyssinia, Nubia, Egypt, Syria, Asia
Minor, the whole of our Europe; that they have overthrown everything,
ransacked everything; there will still remain a few bakers, a few
cobblers, a few tailors, a few carpenters: the necessary arts will
survive; only luxury will be annihilated. It is what was seen at the
fall of the Roman Empire; the art of writing even became very rare;
almost all those which contributed to the comfort of life were reborn
only long after. We invent new ones every day.

From all this one can at bottom conclude nothing against the antiquity
of the globe. For, supposing even that an influx of barbarians had made
us lose entirely all the arts even to the arts of writing and making
bread; supposing, further, that for ten years past we had no bread,
pens, ink and paper; the land which has been able to subsist for ten
years without eating bread and without writing its thoughts, would be
able to pass a century, and a hundred thousand centuries without these
aids.

It is quite clear that man and the other animals can exist very well
without bakers, without novelists, and without theologians, witness the
whole of America, witness three quarters of our continent.

The newness of the arts among us does not therefore prove the newness of
the globe, as was claimed by Epicurus, one of our predecessors in
reverie, who supposed that by chance the eternal atoms in declining, had
one day formed our earth. Pomponace said: \enquote{\textit{Se il mondo non è eterno,
per tutti santi è molto vecchio.}}




\vword{ASTROLOGY}


Astrology may rest on better foundations than Magic. For if no one has
seen either Goblins, or Lemures, or Dives, or Peris, or Demons, or
Cacodemons, the predictions of astrologers have often been seen to
succeed. If of two astrologers consulted on the life of a child and on
the weather, one says that the child will live to manhood, the other
not; if one announces rain, and the other fine weather, it is clear that
one of them will be a prophet.

The great misfortune of the astrologers is that the sky has changed
since the rules of the art were established. The sun, which at the
equinox was in Aries in the time of the Argonauts, is to-day in Taurus;
and the astrologers, to the great ill-fortune of their art, to-day
attribute to one house of the sun what belongs visibly to another.
However, that is not a demonstrative reason against astrology. The
masters of the art deceive themselves; but it is not demonstrated that
the art cannot exist.

There is no absurdity in saying: Such and such a child is born in the
waxing of the moon, during stormy weather, at the rising of such and
such star; his constitution has been feeble, and his life unhappy and
short; which is the ordinary lot of poor constitutions: this child, on
the contrary, was born when the moon was full, the sun strong, the
weather calm, at the rising of such and such star; his constitution has
been good, his life long and happy. If these observations had been
repeated, if they had been found accurate, experience would have been
able after some thousands of years to form an art which it would have
been difficult to doubt: one would have thought, with some likelihood,
that men are like trees and vegetables which must be planted and sown
only in certain seasons. It would have been of no avail against the
astrologers to say: My son was born at a fortunate time, and
nevertheless died in his cradle; the astrologer would have answered: It
often happens that trees planted in the proper season perish; I answered
to you for the stars, but I did not answer for the flaw of conformation
you communicated to your child. Astrology operates only when no cause
opposes itself to the good the stars can do.

One would not have succeeded better in discrediting the astrologer by
saying: Of two children who were born in the same minute, one has been
king, the other has been only churchwarden of his parish; for the
astrologer could very well have defended himself by pointing out that
the peasant made his fortune when he became churchwarden, as the prince
when he became king.

And if one alleged that a bandit whom Sixtus V. had hanged was born at
the same time as Sixtus V., who from a pig-herd became Pope, the
astrologers would say one had made a mistake of a few seconds, and that
it is impossible, according to the rules, for the same star to give the
triple crown and the gibbet. It is then only because a host of
experiences belied the predictions, that men perceived at last that the
art was illusory; but before being undeceived, they were long credulous.

One of the most famous mathematicians in Europe, named Stoffler, who
flourished in the fifteenth and sixteenth centuries, and who long worked
at the reform of the calendar, proposed at the Council of Constance,
foretold a universal flood for the year 1524. This flood was to arrive
in the month of February, and nothing is more plausible; for Saturn,
Jupiter and Mars were then in conjunction in the sign of Pisces. All the
peoples of Europe, Asia and Africa, who heard speak of the prediction,
were dismayed. Everyone expected the flood, despite the rainbow. Several
contemporary authors record that the inhabitants of the maritime
provinces of Germany hastened to sell their lands dirt cheap to those
who had most money, and who were not so credulous as they. Everyone
armed himself with a boat as with an ark. A Toulouse doctor, named
Auriol, had a great ark made for himself, his family and his friends;
the same precautions were taken over a large part of Italy. At last the
month of February arrived, and not a drop of water fell: never was month
more dry, and never were the astrologers more embarrassed. Nevertheless
they were not discouraged, nor neglected among us; almost all princes
continued to consult them.

I have not the honour of being a prince; but the celebrated Count of
Boulainvilliers and an Italian, named Colonne, who had much prestige in
Paris, both foretold that I should die infallibly at the age of
thirty-two. I have been so malicious as to deceive them already by
nearly thirty years, wherefore I humbly beg their pardon.




\vword{ATHEISM}


SECTION I

OF THE COMPARISON SO OFTEN MADE BETWEEN ATHEISM AND IDOLATRY

It seems to me that in the \enquote{Encyclopedic Dictionary} the opinion of the
Jesuit Richeome, on atheists and idolaters, has not been refuted as
strongly as it might have been; opinion held formerly by St. Thomas, St.
Gregory of Nazianze, St. Cyprian and Tertullian, opinion that Arnobius
set forth with much force when he said to the pagans: \enquote{Do you not blush
to reproach us with despising your gods, and is it not much more proper
to believe in no God at all, than to impute to them infamous
actions?}\footnote{Arnobius, \textit{Adversus Gentes.}, lib. v.} opinion established long before by Plutarch, who says \enquote{that
he much prefers people to say there is no Plutarch, than to say--'There
is an inconstant, choleric, vindictive Plutarch'}\footnote{\textit{Of Superstition}, by Plutarch.}; opinion
strengthened finally by all the effort of Bayle's dialectic.


Here is the ground of dispute, brought to fairly dazzling light by the
Jesuit Richeome, and rendered still more plausible by the way Bayle has
turned it to account.\footnote{See Bayle, \textit{Continuation of Divers Thoughts}, par. 77, art. XIII.}
\enquote{There are two porters at the door of a house; they are asked: 'Can one
speak to your master?' 'He is not there,' answers one. 'He is there,'
answers the other, 'but he is busy making counterfeit money, forged
contracts, daggers and poisons, to undo those who have but accomplished
his purposes.' The atheist resembles the first of these porters, the
pagan the other. It is clear, therefore, that the pagan offends the
Deity more gravely than does the atheist.}

With Father Richeome's and even Bayle's permission, that is not at all
the position of the matter. For the first porter to resemble the
atheists, he must not say--\enquote{My master is not here}: he should say--\enquote{I
have no master; him whom you claim to be my master does not exist; my
comrade is a fool to tell you that he is busy compounding poisons and
sharpening daggers to assassinate those who have executed his caprices.
No such being exists in the world.}

Richeome has reasoned, therefore, very badly. And Bayle, in his somewhat
diffuse discourses, has forgotten himself so far as to do Richeome the
honour of annotating him very malapropos.

Plutarch seems to express himself much better in preferring people who
affirm there is no Plutarch, to those who claim Plutarch to be an
unsociable man. In truth, what does it matter to him that people say he
is not in the world? But it matters much to him that his reputation be
not tarnished. It is not thus with the Supreme Being.

Plutarch even does not broach the real object under discussion. It is
not a question of knowing who offends more the Supreme Being, whether it
be he who denies Him, or he who distorts Him. It is impossible to know
otherwise than by revelation, if God is offended by the empty things men
say of Him.

Without a thought, philosophers fall almost always into the ideas of the
common herd, in supposing God to be jealous of His glory, to be
choleric, to love vengeance, and in taking rhetorical figures for real
ideas. The interesting subject for the whole universe, is to know if it
be not better, for the good of all mankind, to admit a rewarding and
revengeful God, who recompenses good actions hidden, and who punishes
secret crimes, than to admit none at all.

Bayle exhausts himself in recounting all the infamies imputed by fable
to the gods of antiquity. His adversaries answer him with commonplaces
that signify nothing. The partisans of Bayle and his enemies have
almost always fought without making contact. They all agree that Jupiter
was an adulterer, Venus a wanton, Mercury a rogue. But, as I see it,
that is not what needs consideration. One must distinguish between
Ovid's Metamorphoses and the religion of the ancient Romans. It is quite
certain that never among the Romans or even among the Greeks, was there
a temple dedicated to Mercury the rogue, Venus the wanton, Jupiter the
adulterer.

The god whom the Romans called \textit{Deus optimus}, very good, very great,
was not reputed to encourage Clodius to sleep with Caesar's wife, or
Cæsar to be King Nicomedes' Sodomite.

Cicero does not say that Mercury incited Verres to steal Sicily,
although Mercury, in the fable, had stolen Apollo's cows. The real
religion of the ancients was that Jupiter, \textit{very good and very just},
and the secondary gods, punished the perjurer in the infernal regions.
Likewise the Romans were long the most religious observers of oaths.
Religion was very useful, therefore, to the Romans. There was no command
to believe in Leda's two eggs, in the changing of Inachus' daughter into
a cow, in the love of Apollo for Hyacinthus.

One must not say therefore that the religion of Numa dishonoured the
Deity. For a long time, therefore, people have been disputing over a
chimera; which happens only too often.

The question is then asked whether a nation of atheists can exist; it
seems to me that one must distinguish between the nation properly so
called, and a society of philosophers above the nation. It is very true
that in every country the populace has need of the greatest curb, and
that if Bayle had had only five or six hundred peasants to govern, he
would not have failed to announce to them the existence of a God,
rewarder and revenger. But Bayle would not have spoken of Him to the
Epicureans who were rich people, fond of rest, cultivating all the
social virtues, and above all friendship, fleeing the embarrassment and
danger of public affairs, in fine, leading a comfortable and innocent
life. It seems to me that in this way the dispute is finished as regards
society and politics.

For entirely savage races, it has been said already that one cannot
count them among either the atheists or the theists. Asking them their
belief would be like asking them if they are for Aristotle or
Democritus: they know nothing; they are not atheists any more than they
are Peripatetics.

In this case, I shall answer that the wolves live like this, and that an
assembly of cannibal barbarians such as you suppose them is not a
society; and I shall always ask you if, when you have lent your money to
someone in your society, you want neither your debtor, nor your
attorney, nor your judge, to believe in God.


OF MODERN ATHEISTS. REASONS OF THE WORSHIPPERS OF GOD

We are intelligent beings: intelligent beings cannot have been formed by
a crude, blind, insensible being: there is certainly some difference
between the ideas of Newton and the dung of a mule. Newton's
intelligence, therefore, came from another intelligence.

When we see a beautiful machine, we say that there is a good engineer,
and that this engineer has excellent judgment. The world is assuredly an
admirable machine; therefore there is in the world an admirable
intelligence, wherever it may be. This argument is old, and none the
worse for that.

All living bodies are composed of levers, of pulleys, which function
according to the laws of mechanics; of liquids which the laws of
hydrostatics cause to circulate perpetually; and when one thinks that
all these beings have a perception quite unrelated to their
organization, one is overwhelmed with surprise.

The movement of the heavenly bodies, that of our little earth round the
sun, all operate by virtue of the most profound mathematical law. How
Plato who was not aware of one of these laws, eloquent but visionary
Plato, who said that the earth was erected on an equilateral triangle,
and the water on a right-angled triangle; strange Plato, who says there
can be only five worlds, because there are only five regular bodies:
how, I say, did Plato, who did not know even spherical trigonometry,
have nevertheless a genius sufficiently fine, an instinct sufficiently
happy, to call God the \enquote{Eternal Geometer,} to feel the existence of a
creative intelligence? Spinoza himself admits it. It is impossible to
strive against this truth which surrounds us and which presses on us
from all sides.

\noindent
REASONS OF THE ATHEISTS

Notwithstanding, I have known refractory persons who say that there is
no creative intelligence at all, and that movement alone has by itself
formed all that we see and all that we are. They tell you brazenly:

\enquote{The combination of this universe was possible, seeing that the
combination exists: therefore it was possible that movement alone
arranged it. Take four of the heavenly bodies only, Mars, Venus, Mercury
and the Earth: let us think first only of the place where they are,
setting aside all the rest, and let us see how many probabilities we
have that movement alone put them in their respective places. We have
only twenty-four chances in this combination, that is, there are only
twenty-four chances against one to bet that these bodies will not be
where they are with reference to each other. Let us add to these four
globes that of Jupiter; there will be only a hundred and twenty against
one to bet that Jupiter, Mars, Venus, Mercury and our globe, will not be
placed where we see them.}

\enquote{Add finally Saturn: there will be only seven hundred and twenty chances
against one, for putting these six big planets in the arrangement they
preserve among themselves, according to their given distances. It is
therefore demonstrated that in seven hundred and twenty throws,
movement alone has been able to put these six principal planets in their
order.}

\enquote{Take then all the secondary bodies, all their combinations, all their
movements, all the beings that vegetate, that live, that feel, that
think, that function in all the globes, you will have but to increase
the number of chances; multiply this number in all eternity, up to the
number which our feebleness calls 'infinity,' there will always be a
unity in favour of the formation of the world, such as it is, by
movement alone: therefore it is possible that in all eternity the
movement of matter alone has produced the entire universe such as it
exists. It is even inevitable that in eternity this combination should
occur. Thus,} they say, \enquote{not only is it possible for the world to be
what it is by movement alone, but it was impossible for it not to be
likewise after an infinity of combinations.}

ANSWER

All this supposition seems to me prodigiously fantastic, for two
reasons; first, that in this universe there are intelligent beings, and
that you would not know how to prove it possible for movement alone to
produce understanding; second, that, from your own avowal, there is
infinity against one to bet, that an intelligent creative cause animates
the universe. When one is alone face to face with the infinite, one
feels very small.

Again, Spinoza himself admits this intelligence; it is the basis of his
system. You have not read it, and it must be read. Why do you want to go
further than him, and in foolish arrogance plunge your feeble reason in
an abyss into which Spinoza dared not descend? Do you realize thoroughly
the extreme folly of saying that it is a blind cause that arranges that
the square of a planet's revolution is always to the square of the
revolutions of other planets, as the cube of its distance is to the cube
of the distances of the others to the common centre? Either the
heavenly bodies are great geometers, or the Eternal Geometer has
arranged the heavenly bodies.

But where is the Eternal Geometer? is He in one place or in all places,
without occupying space? I have no idea. Is it of His own substance that
He has arranged all things? I have no idea. Is He immense without
quantity and without quality? I have no idea. All that I know is that
one must worship Him and be just.

\noindent
NEW OBJECTION OF A MODERN ATHEIST\footnote{See, for this objection, Maupertuis' Essay on Cosmology, first part.}

Can one say that the parts of animals conform to their needs: what are
these needs? preservation and propagation. Is it astonishing then that,
of the infinite combinations which chance has produced, there has been
able to subsist only those that have organs adapted to the nourishment
and continuation of their species? have not all the others perished of
necessity?

\noindent
ANSWER

This objection, oft-repeated since Lucretius, is sufficiently refuted by
the gift of sensation in animals, and by the gift of intelligence in
man. How should combinations \enquote{which chance has produced,} produce this
sensation and this intelligence (as has just been said in the preceding
paragraph)? Without any doubt the limbs of animals are made for their
needs with incomprehensible art, and you are not so bold as to deny it.
You say no more about it. You feel that you have nothing to answer to
this great argument which nature brings against you. The disposition of
a fly's wing, a snail's organs suffices to bring you to the ground.

\noindent
MAUPERTUIS' OBJECTION

Modern natural philosophers have but expanded these so-called arguments,
often they have pushed them to trifling and indecency. They have found
God in the folds of the skin of the rhinoceros: one could, with equal
reason, deny His existence because of the tortoise's shell.

\noindent
ANSWER

What reasoning! The tortoise and the rhinoceros, and all the different
species, are proof equally in their infinite variety of the same cause,
the same design, the same aim, which are preservation, generation and
death.

There is unity in this infinite variety; the shell and the skin bear
witness equally. What! deny God because shell does not resemble leather!
And journalists have been prodigal of eulogies about these ineptitudes,
eulogies they have not given to Newton and Locke, both worshippers of
the Deity who spoke with full knowledge.

\noindent
MAUPERTUIS' OBJECTION

Of what use are beauty and proportion in the construction of the snake?
They may have uses, some say, of which we are ignorant. At least let us
be silent then; let us not admire an animal which we know only by the
harm it does.

\noindent
ANSWER

And be you silent too, seeing that you cannot conceive its utility any
more than I can; or avow that in reptiles everything is admirably
proportioned.

Some are venomous, you have been so yourself. Here there is question
only of the prodigious art which has formed snakes, quadrupeds, birds,
fish and bipeds. This art is sufficiently evident. You ask why the snake
does harm? And you, why have you done harm so many times? Why have you
been a persecutor? which is the greatest of all crimes for a
philosopher. That is another question, a question of moral and physical
ill. For long has one asked why there are so many snakes and so many
wicked men worse than snakes. If flies could reason, they would complain
to God of the existence of spiders; but they would admit what Minerva
admitted about Arachne, in the fable, that she arranges her web
marvellously.

One is bound therefore to recognize an ineffable intelligence which even
Spinoza admitted. One must agree that this intelligence shines in the
vilest insect as in the stars. And as regards moral and physical ill,
what can one say, what do? console oneself by enjoying physical and
moral good, in worshipping the Eternal Being who has made one and
permitted the other.

One more word on this subject. Atheism is the vice of a few intelligent
persons, and superstition is the vice of fools. But rogues! what are
they? rogues.


SECTION II

Let us say a word on the moral question set in action by Bayle, to know
\enquote{if a society of atheists could exist?} Let us mark first of all in this
matter what is the enormous contradiction of men in this dispute; those
who have risen against Bayle's opinion with the greatest ardour; those
who have denied with the greatest insults the possibility of a society
of atheists, have since maintained with the same intrepidity that
atheism is the religion of the government of China.

Assuredly they are quite mistaken about the Chinese government; they had
but to read the edicts of the emperors of this vast country to have
seen that these edicts are sermons, and that everywhere there is mention
of the Supreme Being, ruler, revenger, rewarder.

But at the same time they are not less mistaken on the impossibility of
a society of atheists; and I do not know how Mr. Bayle can have
forgotten one striking example which was capable of making his cause
victorious.

In what does a society of atheists appear impossible? It is that one
judges that men who had no check could never live together; that laws
can do nothing against secret crimes; that a revengeful God who punishes
in this world or the other the wicked who have escaped human justice is
necessary.

The laws of Moses, it is true, did not teach a life to come, did not
threaten punishments after death, did not teach the first Jews the
immortality of the soul; but the Jews, far from being atheists, far from
believing in avoiding divine vengeance, were the most religious of all
men. Not only did they believe in the existence of an eternal God, but
they believed Him always present among them; they trembled lest they be
punished in themselves, in their wives, in their children, in their
posterity, even unto the fourth generation; this curb was very potent.

But, among the Gentiles, many sects had no curb; the sceptics doubted
everything: the academicians suspended judgment on everything; the
Epicureans were persuaded that the Deity could not mix Himself in the
affairs of men; and at bottom, they admitted no Deity. They were
convinced that the soul is not a substance, but a faculty which is born
and which perishes with the body; consequently they had no yoke other
than morality and honour. The Roman senators and knights were veritable
atheists, for the gods did not exist for men who neither feared nor
hoped anything from them. The Roman senate in the time of Cæsar and
Cicero, was therefore really an assembly of atheists.

That great orator, in his harangue for Cluentius, says to the whole
senate in assembly: \enquote{What ill does death do him? we reject all the inept
fables of the nether regions: of what then has death deprived him? of
nothing but the consciousness of suffering.}

Does not Cæsar, the friend of Cataline, wishing to save his friend's
life against this same Cicero, object to him that to make a criminal die
is not to punish him at all, that death \textit{is nothing}, that it is merely
the end of our ills, that it is a moment more happy than calamitous? And
do not Cicero and the whole senate surrender to these reasons? The
conquerors and the legislators of the known universe formed visibly
therefore a society of men who feared nothing from the gods, who were
real atheists.

Further on Bayle examines whether idolatry is more dangerous than
atheism, if it is a greater crime not to believe in the Deity than to
have unworthy opinions thereof: in that he is of Plutarch's opinion; he
believes it is better to have no opinion than to have a bad opinion; but
with all deference to Plutarch, it was clearly infinitely better for the
Greeks to fear Ceres, Neptune and Jupiter, than to fear nothing at all.
The sanctity of oaths is clearly necessary, and one should have more
confidence in those who believe that a false oath will be punished, than
in those who think they can make a false oath with impunity. It is
indubitable that in a civilized town, it is infinitely more useful to
have a religion, even a bad one, than to have none at all.

It looks, therefore, that Bayle should have examined rather which is the
more dangerous, fanaticism or atheism. Fanaticism is certainly a
thousand times more deadly; for atheism inspires no bloody passion,
whereas fanaticism does: atheism is not opposed to crime, but fanaticism
causes crimes to be committed. Fanatics committed the massacres of St.
Bartholomew. Hobbes passed for an atheist; he led a tranquil and
innocent life. The fanatics of his time deluged England, Scotland and
Ireland with blood. Spinoza was not only atheist, but he taught atheism;
it was not he assuredly who took part in the judicial assassination of
Barneveldt; it was not he who tore the brothers De Witt in pieces, and
who ate them grilled.

The atheists are for the most part impudent and misguided scholars who
reason badly, and who not being able to understand the creation, the
origin of evil, and other difficulties, have recourse to the hypothesis
of the eternity of things and of inevitability.

The ambitious, the sensual, have hardly time for reasoning, and for
embracing a bad system; they have other things to do than comparing
Lucretius with Socrates. That is how things go among us.

That was not how things went with the Roman senate which was almost
entirely composed of atheists in theory and in practice, that is to say,
who believed in neither a Providence nor a future life; this senate was
an assembly of philosophers, of sensualists and ambitious men, all very
dangerous, who ruined the republic. Epicureanism existed under the
emperors: the atheists of the senate had been rebels in the time of
Sylla and Cæsar: under Augustus and Tiberius they were atheist slaves.

I would not wish to have to deal with an atheist prince, who would find
it to his interest to have me ground to powder in a mortar: I should be
quite sure of being ground to powder. If I were a sovereign, I would not
wish to have to deal with atheist courtiers, whose interest it would be
to poison me: I should have to be taking antidotes every day. It is
therefore absolutely necessary for princes and for peoples, that the
idea of a Supreme Being, creator, ruler, rewarder, revenger, shall be
deeply engraved in people's minds.

Bayle says, in his \enquote{Thoughts on the Comets,} that there are atheist
peoples. The Caffres, the Hottentots, the Topinambous, and many other
small nations, have no God: they neither deny nor affirm; they have
never heard speak of Him; tell them that there is a God: they will
believe it easily; tell them that everything happens through the nature
of things; they will believe you equally. To claim that they are
atheists is to make the same imputation as if one said they are
anti-Cartesian; they are neither for nor against Descartes. They are
real children; a child is neither atheist nor deist, he is nothing.

What conclusion shall we draw from all this? That atheism is a very
pernicious monster in those who govern; that it is also pernicious in
the persons around statesmen, although their lives may be innocent,
because from their cabinets it may pierce right to the statesmen
themselves; that if it is not so deadly as fanaticism, it is nearly
always fatal to virtue. Let us add especially that there are less
atheists to-day than ever, since philosophers have recognized that there
is no being vegetating without germ, no germ without a plan, etc., and
that wheat comes in no wise from putrefaction.

Some geometers who are not philosophers have rejected final causes, but
real philosophers admit them; a catechist proclaims God to the children,
and Newton demonstrates Him to the learned.

If there are atheists, whom must one blame, if not the mercenary tyrants
of souls, who, making us revolt against their knaveries, force a few
weak minds to deny the God whom these monsters dishonour. How many times
have the people's leeches brought oppressed citizens to the point of
revolting against their king!

Men fattened on our substance cry to us: \enquote{Be persuaded that a she-ass
has spoken; believe that a fish has swallowed a man and has given him up
at the end of three days safe and sound on the shore; have no doubt that
the God of the universe ordered one Jewish prophet to eat excrement
(Ezekiel), and another prophet to buy two whores and to make with them
sons of whoredom (Hosea). These are the very words that the God of truth
and purity has been made to utter; believe a hundred things either
visibly abominable or mathematically impossible; unless you do, the God
of pity will burn you, not only during millions of thousands of millions
of centuries in the fire of hell, but through all eternity, whether you
have a body, whether you have not.}

These inconceivable absurdities revolt weak and rash minds, as well as
wise and resolute minds. They say: \enquote{Our masters paint God to us as the
most insensate and the most barbarous of all beings; therefore there is
no God;} but they should say: therefore our masters attribute to God
their absurdities and their furies, therefore God is the contrary of
what they proclaim, therefore God is as wise and as good as they make
him out mad and wicked. It is thus that wise men account for things. But
if a bigot hears them, he denounces them to a magistrate who is a
watchdog of the priests; and this watchdog has them burned over a slow
fire, in the belief that he is avenging and imitating the divine majesty
he outrages.

\vword{AUTHORITY}

Wretched human beings, whether you wear green robes, turbans, black
robes or surplices, cloaks and neckbands, never seek to use authority
where there is question only of reason, or consent to be scoffed at
throughout the centuries as the most impertinent of all men, and to
suffer public hatred as the most unjust.

A hundred times has one spoken to you of the insolent absurdity with
which you condemned Galileo, and I speak to you for the hundred and
first, and I hope you will keep the anniversary of it for ever; I desire
that there be graved on the door of your Holy Office:

\enquote{Here seven cardinals, assisted by minor brethren, had the master of
thought in Italy thrown into prison at the age of seventy; made him fast
on bread and water because he instructed the human race, and because
they were ignorant.}

There was pronounced a sentence in favour of Aristotle's categories, and
there was decreed learnedly and equitably the penalty of the galleys for
whoever should be sufficiently daring as to have an opinion different
from that of the Stagyrite, whose books were formerly burned by two
councils.

Further on a faculty, which had not great faculties, issued a decree
against innate ideas, and later a decree for innate ideas, without the
said faculty being informed by its beadles what an idea is.

In the neighbouring schools judicial proceedings were instituted against
the circulation of the blood.

An action was started against inoculation, and parties have been
subpoenaed.

At the Customs of thought twenty-one folio volumes were seized, in which
it was stated treacherously and wickedly that triangles always have
three angles; that a father is older than his son; that Rhea Silvia lost
her virginity before giving birth to her child, and that flour is not an
oak leaf.

In another year was judged the action: \textit{Utrum chimera bombinans in vacuo possit comedere secundas intentiones}, and was decided in the
affirmative.

In consequence, everyone thought themselves far superior to Archimedes,
Euclid, Cicero, Pliny, and strutted proudly about the University
quarter.

\vword{AUTHORS}


Author is a generic name which can, like the name of all other
professions, signify good or bad, worthy of respect or ridicule, useful
and agreeable, or trash for the wastepaper-basket.
\begin{center}
       *       *       *       *       *
\end{center}
We think that the author of a good work should refrain from three
things--from putting his name, save very modestly, from the epistle
dedicatory, and from the preface. Others should refrain from a
fourth--that is, from writing.
\begin{center}
       *       *       *       *       *
\end{center}
Prefaces are another stumbling-block. \enquote{The 'I,'} said Pascal, \enquote{is
hateful.} Speak as little of yourself as possible; for you must know
that the reader's self-esteem is as great as yours. He will never
forgive you for wanting to condemn him to have a good opinion of you. It
is for your book to speak for you, if it comes to be read by the crowd.
\begin{center}
       *       *       *       *       *
\end{center}
If you want to be an author, if you want to write a book; reflect that
it must be useful and new, or at least infinitely agreeable.
\begin{center}
       *       *       *       *       *
\end{center}
If an ignoramus, a pamphleteer, presumes to criticize without
discrimination, you can confound him; but make rare mention of him, for
fear of sullying your writings.
\begin{center}
       *       *       *       *       *
\end{center}
If you are attacked as regards your style, never reply; it is for your
work alone to make answer.
\begin{center}
       *       *       *       *       *
\end{center}
Someone says you are ill, be content that you are well, without wanting
to prove to the public that you are in perfect health. And above all
remember that the public cares precious little whether you are well or
ill.
\begin{center}
       *       *       *       *       *
\end{center}
A hundred authors make compilations in order to have bread, and twenty
pamphleteers make excerpts from these compilations, or apology for them,
or criticism and satire of them, also with the idea of having bread,
because they have no other trade. All these persons go on Friday to the
police lieutenant of Paris to ask permission to sell their rubbish. They
have audience immediately after the strumpets who do not look at them
because they know that these are underhand dealings.\footnote{When Voltaire was writing, it was the police lieutenant of Paris who
had, under the chancellor, the inspection of books: since then, a part
of his department has been taken from him. He has kept only the
inspection of theatrical plays and works below those on printed sheets.
The detail of this part is immense. In Paris one is not permitted to
print that one has lost one's dog, unless the police are assured that in
the poor beast's description there is no proposition contrary to
morality and religion (1819).}

\begin{center}
       *       *       *       *       *
\end{center}
Real authors are those who have succeeded in one of the real arts, in
epic poetry, in tragedy or comedy, in history or philosophy, who have
taught men or charmed them. The others of whom we have spoken are, among
men of letters, what wasps are among birds.