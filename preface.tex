\chapter*{PREFACE}
\addcontentsline{toc}{section}{Preface}
 \setcounter{page}{1}
This book does not demand continuous reading; but at whatever place one
opens it, one will find matter for reflection. The most useful books are
those of which readers themselves compose half; they extend the thoughts
of which the germ is presented to them; they correct what seems
defective to them, and they fortify by their reflections what seems to
them weak.

It is only really by enlightened people that this book can be read; the
ordinary man is not made for such knowledge; philosophy will never be
his lot. Those who say that there are truths which must be hidden from
the people, need not be alarmed; the people do not read; they work six
days of the week, and on the seventh go to the inn. In a word,
philosophical works are made only for philosophers, and every honest man
must try to be a philosopher, without pluming himself on being one.

This alphabet is extracted from the most estimable works which are not
commonly within the reach of the many; and if the author does not always
mention the sources of his information, as being well enough known to
the learned, he must not be suspected of wishing to take the credit for
other people's work, because he himself preserves anonymity, according
to this word of the Gospel: \enquote{Let not thy left hand know what thy right hand doeth.}

